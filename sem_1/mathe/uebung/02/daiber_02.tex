\documentclass{article}
\title{Mathe Übung 2}
\author{Nick Daiber}
\usepackage{amsmath}
\usepackage{amssymb}

\begin{document}
    \maketitle
    \section{}
        \begin{tabular}{c|c|c|c|c}
            \textbf{P}& \textbf{Q}&
            \textbf{$\neg P\Rightarrow Q$}& \textbf{$\neg P\Rightarrow \neg Q$}&
            \textbf{$(\neg P\Rightarrow Q)\land(\neg P\Rightarrow \neg Q)\Rightarrow P$}\\
            0&0&0&1&0\\
            0&1&1&0&0\\
            1&0&1&1&1\\
            1&1&1&1&1\\
        \end{tabular}
    \section{}
        \subsection{}
            \paragraph{Induktionsvorraussetzung}
                \begin{align*}
                    \sum_{k=1}^1 k^3 = 1 = \frac{1^2(1+1)^2}{4}
                \end{align*}
            \paragraph{Induktionsannahme}
                \begin{align*}
                    \sum_{k=1}^n k^3 = \frac{n^2(n+1)^2}{4}
                \end{align*}
            \paragraph{Induktionsschritt}
                \begin{align*}
                    \sum_{k=1}^{n+1} k^3 
                    &= \sum_{k=1}^{n} k^3 + (n+1)^3\\
                    &=\frac{n^2(n+1)^2}{4} + (n+1)^2(n+1)\\
                    &=\frac{n^2(n+1)^2}{4} + \frac{4(n+1)^2(n+1)}{4}\\
                    &=\frac{(n+1)^2(n^2+4n+4)}{4}\\
                    &=\frac{(n+1)^2(n+2)^2}{4} \blacksquare
                \end{align*}

        \subsection{}
            \paragraph{Induktionsvorraussetzung}
                $x^1-y^1$ ist von $x-y$ trivial teilbar
            \paragraph{Induktionsannahme}
                $x^n-y^n$ ist von $x-y$ für ein $n\in\mathbb N$ teilbar
            \paragraph{Induktionsschritt}
                \begin{align*}
                    x^{n+1}-y^{n+1} &= x^{n+1} -xy^n+xy^n -y^{n+1}\\
                    &= x^n x-xy^n+xy^n-y^ny\\
                    &=x\underbrace{(x^n-y^n)}_{x-y\text{ teilt}}+
                    \underbrace{y^n(x-y)}_{x-y\text{ teilt}}\blacksquare
                \end{align*}
        \subsection{}
            Es wird zunächst bewiesen, dass $n^2>2n+1$ für $n \geq 5$
            \paragraph{Induktionsvorraussetzung}
                $5^2=25 \geq 2\cdot 5+1$
            \paragraph{Induktionsannahme}
                $n^2\geq 2n+1$ gilt für ein $n\geq 5 \in \mathbb N$
            \paragraph{Induktionsschritt}
                \begin{align*}
                    (n+1)^2&=n^2+n+1\\
                    &\geq 5n+n+1=6n+1\\
                    &> 2n+3 = 2(n+1)+1
                \end{align*}

            Nun wird $2^n>n^2$ bewiesen.
            \paragraph{Induktionsvorraussetzung}
                $2^5=32>25=5^2$
            \paragraph{Induktionsannahme}
                $2^n> n^2$ gilt für ein $n\geq 5 \in \mathbb N$
            \paragraph{Induktionsschritt}
                \begin{align*}
                    2^{(n+1)}&=2\cdot 2^n\\
                    &> 2 \cdot n^2\\
                    &> n^2 + 2n+1 = (n+1)^2 \blacksquare
                \end{align*}
    \section{}
    \section{}
    \section{}
    \begin{align*}
        r_0 = 29&= 1\cdot 17 + 12\\
        r_1 = 17&= 1\cdot 12 + 5\\
        r_2 = 12&= 2\cdot 5 + 2\\
        r_3 = 5&= 2\cdot 2 + 1\\
        r_4 = 2&= 2\cdot 1 + 0
    \end{align*}
    \begin{align*}
        r_0 = 713&= 1\cdot 552 + 161\\
        r_1 = 552&= 3\cdot 161 + 69\\
        r_2 = 161&= 2\cdot 69 + 23\\
        r_3 = 69&= 3\cdot 23 + 0
    \end{align*}
    \begin{align*}
        r_0 = 11253&= 4\cdot 2607 + 825\\
        r_1 = 2607&= 3\cdot 825 + 132\\
        r_2 = 825&= 6\cdot 132 + 33\\
        r_3 = 132&= 4\cdot 33 + 0
    \end{align*}

\end{document}