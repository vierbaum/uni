\documentclass{article}
\usepackage{amsmath}
\usepackage{amssymb}
\usepackage{circledsteps}

\title{Übung 1}
\author{Nick Daiber}

\begin{document}
    \maketitle
    \section{}
    Aufgabe 1. (V) Ein Mann ist in einem Keller gefangen. Nach einer
    kurzen Suche findet er drei
    Türen. Hinter einer der Türen ist ein Weg in die Freiheit.
    Hinter den anderen zwei Türen ist
    jedoch ein b¨oser Feuer speiender Drache. An jeder Tür hängt ein Zettel
    mit einem Hinweis auf
    ihren Inhalt. Mindestens einer der drei Hinweise ist wahr und
    mindestens einer von ihnen ist falsch.
    Die Hinweise lauten:
    Tür führt in die Freiheit.
    Tür 2: Es befindet sich ein Drache hinter dieser Tür.
    Tür 3: Es befindet sich ein Drache hinter der Tür 2.
    Welche Tür sollte der Mann öffnen, um frei zu kommen? Begründen Sie Ihre
    Antwort.

    Führe Tür 1 in die Freiheit, dann gilt $\Circled{1}$ wahr
    und $\Circled{1} \Rightarrow \Circled{1} \land \Circled{2} \land \Circled{3}$\\
    Führe Tür 2 in die Freiheit, dann gilt 
    $\neg\Circled{1}\land \neg \Circled{2} \land \neg \Circled{3}$\\
    Führe Tür 3 in die Freiheit, dann gilt $\neg\Circled{1}\land \Circled{2} \land \Circled{3}$\\
    Da mindestens eine Aussage falsch und eine wahr sein muss, können fallen Tür 1,2 weg.

    \section{}
    A:= Anna hat die Prüfung bestanden.\\
    B:= Britta hat die Prüfung bestanden.\\
    C:= Carlo hat die Prüfung bestanden.\\
    \subsection{}
    $C\land\neg A\land \neg B$
    \subsection{}
    $\neg A \land B \land C$
    \subsection{}
    $(A \land \neg B \land \neg C) \lor (B \land \neg A \land \neg C) \lor
    (C \land \neg A \land \neg B)$
    \subsection{}
    $A \lor B \lor C$
    \subsection{}
    $(A \land B) \lor (A \land C) \lor (B \land C)$
    \subsection{}
    $\neg (A \land B \land C)$
    \subsection{}
    $(A\land B \land \neg C) \lor (B \land C \land \neg A)
    \lor (A \land C \land \neg B)$
    \section{}
    \subsection{}
    \begin{align*}
        (A\cup B)^c &= (\{z\in X | z \in A \lor z \in B\})^C\\
        &= X \setminus \{z\in X | z \in A \lor z \in B\}\\
        &=\{y \in X | y \notin \{z\in X | z \in A \lor z \in B\}\}\\
        &=\{y \in X | y \notin A \land y \notin B\}\\
        &=\{y \in X | y \in \{z \in X | z \notin A\} \land y \in \{z \in X | \notin B\}\}\\
        &=\{y \in X | y \notin A\} \cap \{y \in X | y \notin B\}\\
        &= A^c \cap B^c\\
        \blacksquare
    \end{align*}
    \subsection{}
    \begin{align*}
        (A\cap B)^c &= X \setminus \{z \in X | z \in A \land z \in B\}\\
        &= \{y \in X | y \notin \{z \in X | z \in A \land z \in B\}\}\\
        &= \{y \in X | y \notin A \lor y \notin B\}\\
        &= \{y \in X | y \in \{z \in X |\notin A\} \lor y \in \{z \in X | z \notin B\} \}\\
        &= A^c \cap B^c
    \end{align*}
    \section{}
    \subsection{}
    $A = \mathbb{Z}\setminus \{7\}$\\
    $B = \{-2, 2, -3, 3, -5, 5, -7, 7, -11, 11, -13, 13, -17, 17, -19, 19, -23, 23, -29, 29, -31, 31\}$\\
    $C = (-35, 35)$\\
    $Sei a^3:=x, b=0$, dann ist $a^3-b^3=x$
    \subsection{}
    $A\cap B = B\setminus \{7\}$\\
    $A\cup B = \mathbb{Z}$\\
    $A\cap B \cap C = B\setminus\{7\}$
    \subsection{}
    Sei $I:=[a;b]$ mit $a,b \in \mathbb{Z}, (a>7\lor b<7)\land b>a \Rightarrow [a;b] \subset A \land \emptyset \subset [a;b]$\\
    Mit $c := \frac{b-a}{10}$ definieren wir $A_0 = \{a+ 0\cdot c, a + 1 \cdot c, a + 2\cdot c, \hdots, a + 10\cdot c\}$\\
    Da es unendlich $a<7$ und $b > 7$ gibt, gibt es unendlich Teilmengen von A 
    und unendlich Teilmengen von A mit genau 11 Elementen.
    \subsection{}
    \begin{enumerate}
        \item{Ja, $\frac{12}{1-7}=-6\in\mathbb Z$}
        \item{Nein, da 1 keine Primzahl ist}
        \item{Ja, da $2,3 \in (-35, 35)$}
        \item{Nein, da $B \subset C$}
        \item{Nein, $C$ ist Überabzählbar unendlich}
        \item{Nein, $A\setminus B$ ist Überabzählbar unendlich}
        \item{Ja, $\emptyset \cup C = C$}
        \item{Nein, $A \cap C = (-35, 35)\setminus \{7\}$}
        \item{Nein, $B\cap C = B$}
        \item{Nein, $A \cup C = \mathbb Z$}
    \end{enumerate}



\end{document}
