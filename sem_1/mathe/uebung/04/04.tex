\documentclass{article}
\usepackage{amsmath}
\usepackage{amsfonts}
\usepackage{amssymb}
\author{Nick Daiber}
\title{Mathe 04}

\begin{document}
\maketitle
\section*{3}
\subsection*{b}
\paragraph{Induktionsannahme\\}
Wir zeigen die Identität zunächst für $n\in\{k\in \mathbb N|k\equiv 1 (\mod 5)\}$
ist $n = 1$, so gilt $n^2 = 1 \equiv 1 \mod 5$\\
\paragraph{Induktions Schritt\\}
$(n+5)^2=n^2+10n+25\equiv n^2 (\mod 5), n^2 \equiv 1(\mod 5)$\\
Da $\equiv$ nach 4.5 transitiv ist.

\paragraph{Induktionsannahme\\}
Danach für $n\in\{k\in \mathbb N|k\equiv 2 (\mod 5)\}$
ist $n = 2$, so gilt $n^2 = 4 \equiv -1 \mod 5$\\
\paragraph{Induktions Schritt\\}
$(n+5)^2=n^2+10n+25\equiv n^2 (\mod 5), n^2 \equiv 2(\mod 5)$\\
Da $\equiv$ nach 4.5 transitiv ist.

\paragraph{Induktionsannahme\\}
für $n\in\{k\in \mathbb N|k\equiv 3 (\mod 5)\}$
ist $n = 3$, so gilt $n^2 = 9 \equiv -1 \mod 5$\\
\paragraph{Induktions Schritt\\}
$(n+5)^2=n^2+10n+25\equiv n^2 (\mod 5), n^2 \equiv 3(\mod 5)$\\
Da $\equiv$ nach 4.5 transitiv ist.
\paragraph{Induktionsannahme\\}
für $n\in\{k\in \mathbb N|k\equiv 4 (\mod 5)\}$
ist $n = 4$, so gilt $n^2 = 16 \equiv 1 \mod 5$\\
\paragraph{Induktions Schritt\\}
$(n+5)^2=n^2+10n+25\equiv n^2 (\mod 5), n^2 \equiv 4(\mod 5)$\\
Da $\equiv$ nach 4.5 transitiv ist.
\paragraph{Induktionsannahme\\}
für $n\in\{k\in \mathbb N|k\equiv 0 (\mod 5)\}$
ist $n = 5$, so gilt $n^2 = 25 \equiv 0 \mod 5$
\paragraph{Induktions Schritt\\}
$(n+5)^2=n^2+10n+25\equiv n^2 (\mod 5), n^2 \equiv 0(\mod 5)$\\
Da $\equiv$ nach 4.5 transitiv ist.
\subsection*{b}
\subsubsection*{$2^{2024}$}
\begin{align*}
    2^4 = 16 &\equiv 3 \mod 13\\
    2^{2024}\mod 13&=(2^4)^{506}\mod 13\\
    &=\overline{3}^{506}\\
    &=\overline{(3\cdot 3\cdot 3)^{168}\cdot 3\cdot 3}\\
    &=\overline{1^{168}\cdot 3\cdot 3}\\
    &=1=r_1
\end{align*}
\subsubsection*{$3^{2024}$}
\begin{align*}
    3^{2024}\mod 13&=\overline{3^{2024}}\\
    &=\overline{(3\cdot 3\cdot 3)^{674}\cdot 3\cdot 3}\\
    &=\overline{1^{674}\cdot 3\cdot 3}\\
    &=9=r_2
\end{align*}
\subsubsection*{$5^{2021}$}
\begin{align*}
    5^{2021}\mod 13 &=(5^{2020}\cdot 5)\mod 13\\
    &=\overline{-1^{1010}}\cdot 5\\
    &=1\cdot 5=5=r_3
\end{align*}
\subsubsection*{$7^{2024}$}
\begin{align*}
    7^3=343&\equiv 5(\mod 13)\\
    7^{2024}\mod 13&=(7^{2021}\cdot 49)\mod 13\\
    &=\overline{5^{674}}\cdot 49\\
    &=\overline{-1^{337}}\cdot 49\\
    &=\overline{-1}\cdot 49\\
    &=3=r_4
\end{align*}
\subsection*{c}
Sei $p$ eine ungerade Zahl, so ist
\begin{align*}
    p\equiv 1(\mod 2)\\
    \Rightarrow p^2\equiv 1(\mod 2)\\
    \Rightarrow 8p^2\equiv 1(\mod 2)\\
    \Rightarrow 8p^2+1\equiv 0(\mod 2)
\end{align*}
Also gilt $2|(8p^2+1)$, wenn $p$ ungerade ist, also ist
$(8p^2+1)$ auch keine Primzahl.

$8\cdot 2^2+1 = 33$ ist eine Primzahl und da 2
die einzige gerade Primzahl ist, gilt $(8p^2+1)$
nur für 2.$\blacksquare$
\end{document}