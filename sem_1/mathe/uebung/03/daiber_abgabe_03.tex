\documentclass{article}
\usepackage{amsmath}
\usepackage{amsfonts}
\usepackage{amssymb}
\author{Nick Daiber}
\title{Mathe 03}

\begin{document}
\maketitle
\section*{2}
\subsection*{a}
\paragraph{Reflexivität}
\begin{align*}
    &a=(n,m)\\
    &nm = nm\\
    &\Leftrightarrow(n,m)\sim(n,m)\\
    &\Leftrightarrow a\sim a
\end{align*}
\paragraph{Symmetrie}
\begin{align*}
    &a\sim b\\
    &\Leftrightarrow(n,m)\sim(x,y)\\
    &\Leftrightarrow ny=xm\\
    &\Leftarrow mx=ny\\
    &\Leftrightarrow(x,y)\sim(n,m)\\
    &\Leftrightarrow b\sim a
\end{align*}
\paragraph{Transitivität\\}
Sei $a\sim b\text{ und }b\sim c$\\
dann gilt

\begin{equation}
    n_a m_b=n_b m_a\Leftrightarrow n_b=\frac{n_a m_b}{m_a}
\end{equation}
\begin{equation}
    n_b m_c=n_c m_b
\end{equation}
\begin{align*}
    &n_b m_c\overset{(1)}{=}\frac{n_a m_b m_c}{m_a}\overset{(2)}{=}n_c m_b\\
    &\Leftrightarrow n_a m_c = n_c m_a\\
    &\Leftrightarrow a\sim c\\\blacksquare
\end{align*}
\subsection*{b}
Sei $a:=(n,m)\in \mathbb Z\times\mathbb N$ beliebig.
Man nehme an, es existiert kein ungekürztes paar in $K(a, R)$\\
Wähle $(x,y)\in K(a, R)$ beliebig, es gilt also
\begin{align*}
    &(n,m)\sim(x,y)\\
    &\Leftrightarrow ny=xm \text{ da x,y nach Annahme nicht Teilerfremd}\\
    &\Leftrightarrow n(ky')=(kx')m\\
    &\Leftrightarrow k(ny')=k(x'm)\\
    &\Leftrightarrow ny'=x'm\\
    &\Leftrightarrow (n,m) \sim (x',y')
\end{align*}
Sind x', y' immernoch nicht Teilerfremd, so wiederhole man bis sie es sind.
Da $a\sim(x',y')\in A$ ist $(x',y')\in K(a,R)$ ist ein Widerspruch zur
Annahme $\blacksquare$
\subsection*{c}
Zunächst wird gezeigt, dass es zu jedem $a\in A$ ein $b\in B$ mit $a\sim b$ gibt.\\
Sei $a\in A beliebig$ so ist $a$ entweder gekürzt und $a\sim a\in B$ oder
$a:=(kn, km)$, mit $k = \text{ggT}(n,m)$.\\
dann gilt $kn m = n km$ also $a\sim (n,m)\in B$.
Als nächstes wird gezeigt, dass es genau ein $b\in B$ gibt mit $a\sim b$.
Dafür wird angenommen, dass Elemente aus $B$ mit $a$ in relation stehen.\\
Wähle $a\in A$ beliebig und $b:=(n,m), b':=(x,y)\in B$ mit $a\sim b$ und $a\sim b'$
\begin{align*}
    \Rightarrow b\sim b'\\
    \Leftrightarrow nx = my\\
    \Leftrightarrow n,x,m,y|nx
\end{align*}
Da $n,m$ und $x,y$ Teilerfremd sind, gilt $(n,m)=(x,y)$.
Es gibt also nur ein $b\in B$, das mit $a$ in relation steht
$\blacksquare$

\end{document}