\documentclass{article}
\usepackage{listings}
\usepackage{amsmath}
\usepackage{amssymb}
\title{PSE Blatt 06 - Gruppe 32}
\author{Marvin Hepp (3759031)\& Nick Daiber (3728224)}
\newcommand{\N}{\mathbb{N}}
\newcommand{\F}{\mathbb{F}}
\begin{document}
\maketitle
\section*{3}
\subsection*{a}
Wir nehmen die Menge aller Felder
$\F \subseteq \N\times\N$
die besuchbar, also ohne Wand.
Dann wählen wir einen Startpunkt $s\in \F$ und einen endpunkt
$t\in \F$.
Als nächstes definieren wir 
$n:\F\rightarrow\F:(c, r)\mapsto\{(c\pm 1, r)|(c,r)\in\F\}\bigcup \{c, r\pm 1|(c,r)\in\F\}$
Dann $p:\F\times\F^k\rightarrow \F^m$ mit 
$p(\omega, v) =\begin{cases}\omega&\omega=t\\
    \omega\circ \min\limits_{\tau\in n(\omega)}\{\tau, p(v\circ\omega)\}&\text{sonst}
\end{cases}\}$.

$n$ ist eine Funktion, die die besuchbaren Nachbarfelder findet
und $p$ ist ein $m$-tupel an Feldern, $\omega$ das Aktuelle Feld und
$v$ ein $k$-tupel der bereits besuchten feldern.

Nun setzten wir $s$ auf \verb|paule.getLocation()| und $t$ auf das Feld mit
Körnern. Demnacht gilt $(s\Rightarrow t)=p(s, \{\})$

\end{document}