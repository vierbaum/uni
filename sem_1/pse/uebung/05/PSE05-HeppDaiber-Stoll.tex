\documentclass{article}
\usepackage{listings}
\title{PSE Blatt 04 - Gruppe 32}
\author{Marvin Hepp (3759031)\& Nick Daiber (3728224)}
\begin{document}
\maketitle

\section*{1}
\subsection*{a}
V ist wahr, wenn Paule den Sonnenaufgang beobachten kann impliziert,
dass er nach Osten Blickt und nicht vor einer Wand steht

W ist wahr, wenn Paule nach Osten blickt und nicht vor einer wand steht oder
nicht den Sonnenaufgang beobachtet (Vereinfacht)

W ist wahr, wenn nicht Paule nicht nach Osten blickt oder vor einer wand steht oder
nicht den Sonnenaufgang beobachtet (unvereinfacht, aber echt nicht schön)
\subsection*{b}
\begin{tabular}{l|l|l|l|l|l|l}
    $X$&$Y$&$\neg X$&$\neg Y$&$\neg X \lor Y$&$\neg(\neg X \lor Y)$&$X\land \neg Y$\\
    \hline
    0&0&1&1&1&0&0\\
    0&1&1&0&1&0&0\\
    1&0&0&1&0&1&1\\
    1&1&0&0&1&0&0\\
\end{tabular}
\subsection*{c}
\begin{lstlisting}
(Z) || (Z && (X && (!Y)))
\end{lstlisting}

Mit Funktionen eingesetzt
\begin{lstlisting}
(!paule.canViewSunrise()) || 
(paule.canViewSunrise() && 
((paule.getDirection() == Direction.EAST) && (!paule.frontIsClear())))
\end{lstlisting}

\newpage
\section*{2}
\subsection*{a}
Strikte Operatoren werten jeden Teil des Ausdrucks aus,
Semistrikte brechen schon zuvor ab bsp 
\begin{lstlisting}
    (x!=0) && 2/x>1
\end{lstlisting}
Führt bei Strikter Auswertung zum Fehler, bei Semistrikter nicht

\subsection*{b}
\subsubsection*{i}
\begin{itemize}
    \item { $E_1$ kann immer Ausgewertet werden, da sowohl 
    \verb|ham| als auch \verb|null| definiert sind und
    miteinander verglichen werden können }
    \item { $E_2$ führt zu einem Fehler, da \verb|null| keine
    Methode \verb|grainAvailible| hat.}
    \item { $E_3$ führt zu einem Fehler, da \verb|null| keine
    Methode \verb|grainAvailible| hat.}
\end{itemize}
\subsubsection*{ii}
\begin{itemize}
    \item { $E_1$ kann immer Ausgewertet werden, da sowohl 
    \verb|ham| als auch \verb|null| definiert sind und
    miteinander verglichen werden können }
    \item { $E_2$ führt zu einem Fehler, da \verb|null| keine
    Methode \verb|grainAvailible| hat.}
    \item{ $E_3$ kann Ausgewertet werden, da nach
    \verb|ham == null (==True)| ``abgebrochen" wird}
\end{itemize}
\subsection*{b}
$E_2$ 
\section*{3}
\subsection*{a}
Das Atribut \verb|name| darf nicht \verb|null| sein.\\
Die Referenz \verb|hamster| darf nicht \verb|numm| sein.\\
Der Wert von \verb|grainNumber| muss eine ganzzahl $\geq$ dem überziehungslimit sein.\\
Der Wert von \verb|maximumOverdraft| muss eine ganzzahl $\leq 0$ sein.

\subsection*{b}
Wenn auf \verb|secondaryHamster| zugegriffen wird, als sei es ein \verb|Hamster|-Objekt
kann es zu problemen kommen, besser wäre beispielsweise einen Vektor zu nutzen, besonders,
da so mehr als nur 2 besitzer angegeben werden können.

\subsection*{c}
\paragraph{Vorbedingungen}
\begin{itemize}
    \item Der Hamster muss mindestens so viele Körner bei sich haben, wie er einzahlen kann.
    \item Der Hamster muss an dem Automaten stehen
\end{itemize}
\paragraph{Nachbedingungen}
\begin{itemize}
    \item Der Kontostand muss um die eingezahlten Körner erhöht worden sein
    \item Der Hamster darf die eingezahlten Körner nicht mehr besitzen.
\end{itemize}

\subsection*{d}
\paragraph{Vorbedingungen}
\begin{itemize}
    \item Der Hamster muss am Automat stehen
    \item Es müssen genug Körner auf dem Konto sein oder das überziehungslimit muss für das abheben genügen
\end{itemize}
\paragraph{Nachbedingungen}
\begin{itemize}
    \item Der Hamster muss die abgehobenen Körner bei sich haben
    \item Der Kontostand muss um die Anzahl der abgehobenen Körner reduziert worden sein
\end{itemize}

\subsection*{e}
\verb|this| ist eine Referenz auf das Objekt dem die methode angehört.
\verb|null| ist eine Referenz auf nichts

\subsection*{f}
Zeile 9: Es wird 3 ausgegeben, da der Bezeichner \verb|firstAccount| seit Zeile 7 auf das 
zuletzt erstelle Konto-objekt zeigt, welches mit 3 Körnern initialisiert wurde.\\
Zeile 10: es wird 35 ausgegeben, da der Bezeichner \verb|secondAccount| auf das zuerst erstellte
Kontoobjekt zeigt und dessen Kontostand in Zeile 8 von 5 auf 35 erhöht wurde (insofern 30 Körner aufgeladen werden konnten).\\
Zeile 11: Es wird nochmal 3 ausgegeben, da der Bezeichner \verb|thirdAccount| seit beginn auf das
zuletzt erstellte Objekt zeigt.

\subsection*{g}
\verb|transferGrainsTo(firstAccount, null, 20)|\\
Es wird versucht die \verb|depositGrains| methode aufzurufen obwohl \verb|tooBankAccount|
auf \verb|null| zeigt.

\subsection*{h}
in offensichtlichen Fällen wie den oben genannten schon. Eine Referenz auf \verb|null| kann aber
auch erst zur laufzeit entstehen, wenn ein Objekt gelösch wurde, eine Referenz auf dieses aber
nicht.

\end{document}