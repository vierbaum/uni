\documentclass{article}
\usepackage{listings}
\usepackage{amsmath}
\usepackage{amssymb}
\title{PSE Blatt 08 - Gruppe 32}
\author{Marvin Hepp (3759031)\& Nick Daiber (3728224)}
\begin{document}
\maketitle
\section*{1}
\subsection*{a}
Die Klasse ist nicht Semantisch korrekt, da sie nur mit dem Parameter numberOfWheels
initialisiert wird. Der Wert numberOfSeats wird dabei nie initialisiert, soll aber in der Methode
getNumberOfSeats() abgefragt werden können. Um die Klasse zu korigieren, kann man den
Konstruktor um einen Parameter vom Typ int erweitern, und die Konstante numberOfSeats
damit initialisieren.
Die Klasse gilt als unveränderlich (immutable), da alle Felder (numberOfWheels und
numberOfSeats) final sind.
\subsection*{b}
Die beiden Felder name und birthDate in Zeile 14 und 15 werden nicht als final deklariert.
Daher sind sie veränderlich. Dafür gibt es in Zeile 54 die Methode setName welche man
aufrufen kann um das Feld “name” einer Instanz der Klasse Person zu verändern. Außerdem
gibt es in Zeile 67 die Methode setBirthDate mit der man das Feld “setBirthDate” ändern kann
kann.
Im Konstruktor wird zudem die Referenz auf ein Objekt der Klasse Date direkt übergeben und
keine Kopie erstellt (Zeile 31). Da Date ein veränderliches Objekt ist, lässt sich dieses von
außen verändern, nachdem es zum erstellen einer Instanz der Klasse Person genutzt wurde.
Zum Beispiel kann mit der getBirthDate Methode die referenz zu dem Objekt der Klasse Date
erhalten und dieses dann verändern.
\section*{2}
\subsection*{a}
Defensives Programmieren sollte wann möglich genutzt werden, da es zu
skalierbareren und einfacher nutzbaren code führt. Hierbei ist besonders
wichtig immer zu überprüfen, ob alle werte im programm an entsprechenden
schnittstellen passen.
Offensives Programmieren ist fehleranfälliger, aber schneller.
Besonders bei "proof-of-concept" fällen, wo es nicht darum geht
"guten" code zu schreiben sondern schnell fertig zu sein ist Offensives
Programmieren hilfreich
\subsection*{f}
Man iteriert über alle Wände und formt eine Menge der vorhandenen Türen.
Dies ist zwar langsam, aber sehr robust und schnell zu implementieren.
\end{document}