\documentclass{article}
\title{PSE Blatt 04 - Gruppe 32}
\author{Marvin Hepp (3759031)\& Nick Daiber (3728224)}
\begin{document}
\section*{2}
\subsection*{a}
Die operation \verb|substring| kopiert die Buchstaben von 
(inklusive) \verb|startIndex| bis (exklusive) 
\verb|endIndex| in einen String.  Dabei muss 
\verb|startIndex|$\geq 0$ und
\verb|endIndex|$\leq |$\verb|string|$|$

Durch Zeile 3 wird nichts direkt ausgegeben, aber 
\verb|result| mit wert \verb|Ronnie| initialisiert.
\subsection*{b}
Die Operation \verb|contains(s)| prüft, ob s ein substring
eines anderen Strings ist. Dabei muss \verb|s| eine
\verb|CharSequence| sein.

Zeile 4 gilt \verb|false| aus, da \verb|ROnnie| mit einem 
großen \verb|O| geschrieben ist.
\subsection*{c}
\begin{enumerate}
    \item \verb|statement.IndexOf("i")|
    \item {\verb|statement.charAt(7)|\\
    Hierbei wird angenommen, dass das erste \verb|P| der
    "nullte" Buchstabe ist, sonst
    \verb|statement.chatAt(8)|}
\end{enumerate}
\subsection*{c}
Da \verb|String| \verb|CharSequence| implementiert.

\end{document}