\documentclass{article}
\title{PSE Übung 3}
\author{Marvin Hepp (3759031)\& Nick Daiber (3728224)}
\begin{document}
    \maketitle
    \section*{1}
    \subsection*{a}
    Wenn wir im Hamster UI einen neuen Hamster erstellen werden einige Schritte ausgeführt. Zuerst
    wird im Heap-Speicher ein neues Objekt der Klasse Hamster allokiert, das heißt, es wird Speicher für
    ein neues Objekt vom Typ Hamster reserviert. Anschließend werden die Attribute des neuen Hamster-
    Objekts wie location, direction und territory auf die gewünschten Startwerte gesetzt und es wird eine
    Referenz auf das neu erzeugte Objekt angelegt, sodass das Objekt in weiteren Methoden verwendet
    werden kann. Abschließend wird der neue Hamster in der Benutzeroberfläche angezeigt.
    \subsection*{b}
    Bei der Operation getDefaultHamster() handelt es sich um eine Abfrage. Dies liegt daran, dass die
    Operation ausschließlich den Standardhamster zurückgibt und dabei kein Objekt erzeugt und auch von
    keinem Objekt der Zustand verändert wird.
    \subsection*{c}
    Wenn man die Operation getDefaultHamster() auf dem territory aufruft, dann wird kein neues Objekt
    erstellt, sondern es wird einem lediglich die Referenz zu dem aktuellen Standarthamster
    zurückgegeben, welcher sich bereits fertig initialisiert im Speicher befindet.

    \newpage
    \section*{3}
    \subsection*{a}
    \paragraph{Abfragen}
    \begin{enumerate}
        \item {Möglichkeit alle Termine an einem bestimmten Datum abzufragen\\
        Parameter: Datum\\
        Rückgabe: Vector an Terminen}
        \item {Möglichkeit alle Termine mit \verb|x| im Namen abzufragen\\
        Parameter: \verb|x|\\
        Rückgabe: Vector an Terminen}
        \item {Möglichkeit aktuelles Datum abzufragen\\
        Parameter: Keine\\
        Rückgabe: Datum (wahrscheinlich eigene Klasse)}
    \end{enumerate}
    \paragraph{Kommandos}
    \begin{enumerate}
        \item {Möglichkeit einen Termine an einem bestimmten Datum zu setzen\\
        Parameter: Datum, Termin\\
        Rückgabe: succes / failure}
        \item {Möglichkeit einen wiederkehrenden Termin zu setzen (bspw. alle 14 Tage)\\
        Parameter: Termin, Erstes Datum, Zeitspanne\\
        Rückgabe: succes / failure}
        \item {Möglichkeit einen bestimmten Termin zu Löschen\\
        Parameter: Termin\\
        Rückgabe: succes / failure}
    \end{enumerate}
    \newpage
    \subsection*{b}
    \begin{verbatim}
        /*
        * This method returns all appointments on a given date
        *
        * @param date - date in question
        * @returns all appointments
        *
        * @requires date be an actual date
        * @ensures all appointments on date returned
        */
    \end{verbatim}
    \begin{verbatim}
        /*
        * This method returns the current date
        *
        * @returns current date
        *
        * @ensures the returned date is valid
        */
    \end{verbatim}
    \begin{verbatim}
        /*
        * This method creates an appointment on a date
        *
        * @param date
        * @param appointment
        *
        * @requires valid date, valid appointment
        * @ensures the appointment is created
        */
    \end{verbatim}
    \begin{verbatim}
        /*
        * This method deletes a date
        *
        * @requires appointment still active
        * @ensures the appointment is deleted
        */
    \end{verbatim}

\end{document}