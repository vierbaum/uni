\documentclass{article}

\begin{document}
\section*{1}
\subsection*{1.1}
$U_5 = U_1$ da Spannung in einer Parallelschaltung geteilt wird.\\
$I_5 = \frac{U_5}{R_5}=\frac{4V}{3.3k\Omega}=1.21mA$\\
$D_6$ leitet, da $U_4 = U_1 = 4V > 0.7V$\\
Annahme $D_7$, $D_2$ leitet, $D_6$ sperrt, dann ist
$U_7 = U_1 \frac{R_4}{R_3 + R_4} - U_1\frac{R_3}{R_3+R_4} $\\ 
$=4V \cdot (\frac{1.65k\Omega}{3.3k\Omega + 1.65k\Omega} - 
\frac{1.65k\Omega}{2 \cdot 1.65k\Omega}) = -0.67V < 0.7V$
Deshalb sperrt $D_7$
\subsection*{1.2}
$U_2 = U_1 = 4V \geq 0.7V$, da Spannung in einer
Parallelschaltung geteilt wird. Somit leitet $D_2$.

$U_3 = U_1 \frac{R_3}{R_3 + R_4} =$
$4V \frac{1.65}{2\cdot 1.65}= 2V$

$U_4 = U_1 \frac{R_4}{R_3 + R_4} =$
$4V \frac{1.65}{2\cdot 1.65}= 2V$

$I_3 = \frac{U_3}{R_3} = \frac{2V}{1.65k\Omega}=1.21mA$

$I_4 = \frac{U_4}{R_4} = \frac{2V}{1.65k\Omega}=1.21mA$

\subsection*{1.3}
$\frac{1}{R_{ges}} = \frac{1}{R_5}+\frac{1}{R_3 + R_4}
= \frac{1}{3.3k\Omega} + \frac{1}{3.3k\Omega}$
$\Rightarrow R_{ges} = 1.65k\Omega$

$I_1 = \frac{U_1}{R_{ges}} = \frac{4V}{1.65k\Omega} = 2.42mA$


$I_2 = \frac{U_2}{R_{34}} = \frac{4V}{3.3k\Omega} = 1.21mA$

$I_7 = \frac{U_7}{R_4} = \frac{0V}{1.65k\Omega} = 0mA$

$I_6 = I_5 = 1.21mA$

\subsection*{1.4}
Zunächst testen, ob $D_7$ leitet.\\
$U_7=U_3\frac{R_3}{R_3+R_4}=4V\frac{1}{2}=2V\geq 0.7$
$D_7$ leitet also.

Da $R_{D_7} = 0\Omega$ wird $R_4$ überbrückt. 

$U_4 = 0V$

$U_3 = U_1$

$I_3=\frac{U_3}{R_3}=\frac{4V}{1.65k\Omega}=2.42mA$

$I_4=\frac{U_4}{R_4}=\frac{0V}{1.65k\Omega}=0A$

\end{document}