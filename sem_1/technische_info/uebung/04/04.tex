
\documentclass{article}
\usepackage{amssymb}
\usepackage{tikz}
\usepackage{circuitikz}
\usetikzlibrary{calc}
\usetikzlibrary{tikzmark}

\begin{document}
\section{}
\begin{tabular}{r|cccc|l}
    0&0&0&0&0&0\\
    1&0&0&0&1&-\\
    2&0&0&1&0&-\\
    3&0&0&1&1&0\\
    4&0&1&0&0&1\\
    5&0&1&0&1&1\\
    6&0&1&1&0&-\\
    7&0&1&1&1&1\\
    8&1&0&0&0&-\\
    9&1&0&0&1&1\\
    10&1&0&1&0&0\\
    11&1&0&1&1&0\\
    12&1&1&0&0&1\\
    13&1&1&0&1&-\\
    14&1&1&1&0&1\\
    15&1&1&1&1&1\\
\end{tabular}
\subsection{}
\begin{tabular}{c|c|c|c}
    0&-&1&1\\
    \hline
    -&0&1&-\\
    \hline
    0&0&1&1\\
    \hline
    -&1&-&1\\
\end{tabular}
\subsection{}
Primimplikanten:
\begin{itemize}
    \item{$a\land(\neg b)$}
    \item $(\neg b)\land d$
    \item $c$
\end{itemize}
Kernimplikanten:
\begin{itemize}
    \item $c$
\end{itemize}
\subsection{}
$f=c\lor (a \land (\neg b))$
\subsection{}
\begin{circuitikz}
    \ctikzset{logic ports = european}
    \draw
    (0,2) node[and port] (and) {}
    (2.5,1) node[or port] (or) {}
    (and.in 1) node[left=.2cm](a) {$a$}
    (and.in 2) node[left = .2cm](b) {$b$}
    (or.in 2) node[left = .2cm](c) {$c$}
    (or.out) node[right = .2cm](f) {$f$}
    (and.out) |- (or.in 1)
    (or.out) |- (f)
    (a) -| (and.in 1)
    (b) -| (and.in 2)
    (c) -| (or.in 2)
    ;
    \node at (and.bin 2)[ocirc,left] {};
\end{circuitikz}

\subsection{}
\begin{tabular}{c|c|c|c}
    1&-&0&0\\
    \hline
    -&1&0&-\\
    \hline
    1&1&0&0\\
    \hline
    -&0&-&0\\
\end{tabular}\\
$f=(b\land (\neg c))\lor ((\neg c)\land (\neg d))$
\subsection{}
$f=(b\lor (\neg d))\land (\neg c)$
\subsection{}
\begin{tabular}{c|c|c|c}
    1&1&0&0\\
    \hline
    1&1&0&0\\
    \hline
    1&1&0&0\\
    \hline
    0&0&0&0\\
\end{tabular}
\subsection{}
\begin{circuitikz}
    \ctikzset{logic ports = european}
    \draw
    (0,2) node[or port] (or) {}
    (2.5,1) node[and port] (and) {}
    (or.in 1) node[left=.2cm](d) {$b$}
    (or.in 2) node[left = .2cm](b) {$d$}
    (and.in 2) node[left = .2cm](c) {$c$}
    (and.out) node[right = .2cm](f) {$f$}
    (or.out) |- (and.in 1)
    (and.out) |- (f)
    (d) -| (or.in 1)
    (b) -| (or.in 2)
    (c) -| (and.in 2)
    ;
    \node at (or.bin 1)[ocirc,left] {};
\end{circuitikz}
\newpage
\section{}
\subsection{}
Implikanten 0. Ordnung

\begin{tabular}{c||cccc}
    &$x_4$&$x_3$&$x_2$&$x_1$\\
    \hline\hline
    0&0&0&0&0\\
    1&0&0&0&1\\
    4&0&1&0&0\\
    \hline\hline
    3&0&0&1&1\\
    5&0&1&0&1\\
    6&0&1&1&0\\
    10&1&0&1&0\\
    \hline\hline
    7&0&1&1&1\\
    11&1&0&1&1\\
    14&1&1&1&0\\
    \hline\hline
    15&1&1&1&1
\end{tabular}

Implikanten 1. Ordnung

\begin{tabular}{c||cccc}
    &$x_4$&$x_3$&$x_2$&$x_1$\\
    \hline\hline
    0,1&0&0&0&-\\
    0,4&0&-&0&0\\
    1,3&0&0&-&1\\
    1,5&0&-&0&1\\
    3,7&0&-&1&1\\
    3,11&-&0&1&1\\
    4,5&0&1&0&-\\
    4,6&0&1&-&0\\
    5,7&0&1&-&1\\
    6,7&0&1&1&-\\
    6,14&-&1&1&0\\
    7,15&-&1&1&1\\
    10,11&1&0&1&-\\
    10,14&1&-&1&0\\
    11,15&1&-&1&1\\
    14,15&1&1&1&-
\end{tabular}

Implikanten 2. Ordnung

\begin{tabular}{c||ccccl}
    &$x_4$&$x_3$&$x_2$&$x_1$\\
    \hline\hline
    0,1,4,5&0&-&0&-&$\checkmark$\\
    1,3,5,7&0&-&-&1&$\checkmark$\\
    3,7,11,15&-&-&1&1&$\checkmark$\\
    4,5,6,7&0&1&-&-&$\checkmark$\\
    6,7,14,15&-&1&1&-&$\checkmark$\\
    10,11,14,15&1&-&1&-&$\checkmark$
\end{tabular}
\subsection{}
\begin{tabular}{c||cccccccl}
    &$m_0$&$m_3$&$m_4$&$m_6$&$m_{10}$&$m_{14}$&$m_{15}$\\
    \hline\hline
    0,1,4,5&\tikzmark{i1l}X&&X&&&&\tikzmark{i1r}&$(\neg x_2)\land (\neg x_4)$\\
    1,3,5,7&&X\\
    3,7,11,15&&X&&&&&X\\
    4,5,6,7&&&X&X\\
    6,7,14,15&&&&X&&X&X\\
    10,11,14,15&\tikzmark{i2l}&&&&X&X&X\tikzmark{i2r}&$x_2\land x_4$
\end{tabular}
\begin{tikzpicture}[overlay, remember picture, shorten >=.5pt, shorten <=.5pt, transform canvas={yshift=.25\baselineskip}]
\draw [-] ({pic cs:i1l}) to ({pic cs:i1r});
\draw [-] ({pic cs:i2l}) to ({pic cs:i2r});
\end{tikzpicture}

$\Rightarrow 0,1,4,5$ und $10,11,14,15$ sind Kernimplikanten

\subsection{}
\begin{tabular}{c||ccl}
    &$m_3$&$m_6$\\
    \hline\hline
    1,3,5,7&X&&$x_1\land (\neg x_4)$\\
    3,7,11,15&X&&$x_1\land x_2$\\
    4,5,6,7&&X&$x_3\land (\neg x_4)$\\
    6,7,14,15&&X&$x_2\land x_3$\\
\end{tabular}

Minimale Summe: $f=((\neg x_2)\land (\neg x_4))\lor (x_2 \land x_4) \lor (x_1 \land x_2) \lor (x_2 \land x_3)$
Die Eindeutigkeit der Summe hängt von der definition der Eindeutigkeit ab.
Die Summe gibt zwar eindeutig an, ob ein term zu gegebenen variablen wahr / falsch ist,
aber die Summe ist nicht eineindeutig definiert, da $1,3,5,7$ und $3,7,11,15$, sowie
$4,5,6,7$ und $6,7,14,15$ dieselbem minterme bedienen, aber keine Zeilendominanz übereinander
haben.
\end{document}