\documentclass{article}
\usepackage{amsmath}
\usepackage{amssymb}

\title{Theo1}
\author{Nick Daiber}

\begin{document}

\maketitle
\section{16.10}
$A\times B = \{(a,b) | a\in A, b\in B\}$\\
$(A\times B) \times C = A \times (B \times C)$\\
$\mathbb N = \mathbb N \cup \{0\}$\\
$2^A = P(A)$

Nichttriviale Teilmenge: $A\subset B \land A \neq B \land A \neg \emptyset$\\

\section{Formale Sprachen (17.10)}
\begin{itemize}
	\item{Alphabet $\Sigma$}
	\item{$\Sigma^*$ alle Wörter}
	\item{$\varepsilon$ leeres wort}
	\item{$\Sigma^k$ Wörter der Länge k (als n-Tupel)}
	\item{$|w|$ die Länge des Wortes}
	\item{$w^n = w\cdot w \cdot \hdots \cdot w$}
\end{itemize}
Monoid ist eine Menge mit einer assoziativen Verknüpfung und einem Neutralen Element
Hier: Verknüpfung = Konkatenation

$\Sigma^*$ ist abzählbar unendlich, die Menge aller Sprachen ist $P(\Sigma^*)$ und ebenfalls
abzählbar unendlich

\subsection{Grammatiken}
\begin{itemize}
	\item {$G=(V,\Sigma,P,S)$}
	\item{$V$ Nichtterminale}
	\item{$\Sigma$ Terminale}
	\item{$P$ Produktionsregeln}
	\item{$S\in V$ Startsymbol}
	\item{Typ 0: Beliebig}
	\item{Typ 1: $|u| < |v|$, Wort wird Länger}
	\item{Typ 2: $(u,v)$ mit $u \in V$}
	\item{Typ 3: $(u,v)$, $v = (\Sigma, v_2 \in V)$ oder $v \in \Sigma$, (max 1 nichtterminal)}
\end{itemize}
\end{document}
