\documentclass{article}
\title{Theo 1 Abgabe 2}
\author{Nick Daiber}
\usepackage{amsmath}
\usepackage{amssymb}
\usepackage{amsfonts}
\usepackage{tikz}
\usetikzlibrary{automata,positioning}
\newcommand{\llang}{\langle\langle}
\newcommand{\rrang}{\rangle\rangle}
\newcommand{\N}{\mathbb{N}}

\begin{document}
\maketitle
\section*{1}
\begin{align*}
    \llang LK\rrang &=
    \{w_1,\hdots,w_n|
    \exists w=a_1\hdots a_n\in L
    \text{ mit }a_i\in\Sigma
    \text{ und }w_i\in {L_a}_i
    \text{ für } i=1,\hdots,n\}\\
    &=\{w_1,\hdots,w_k,w_{k+1},\hdots,w_n|
    \exists w=a_1\hdots a_n\in L
    \text{ mit }a_i\in\Sigma
    \text{ und }w_i\in {L_a}_i
    \text{ für } i=1,\hdots,n\}\\
    &=\{v_1,\hdots,v_k,u_{1},\hdots,u_{n-k}|
    \exists u,v=a_1\hdots a_n\in L
    \text{ mit }a_i\in\Sigma
    \text{ und }u_i,v_i\in {L_a}_i
    \text{ für } i=1,\hdots,n\}\\
    &=
    \{v_1,\hdots,v_k|
    \exists v=a_1\hdots a_n\in L
    \text{ mit }a_i\in\Sigma
    \text{ und }v_i\in {L_a}_i
    \text{ für } i=1,\hdots,n\}\\
    &\{u_{1},\hdots,u_{n-k}|
    \exists u=a_1\hdots a_n\in L
    \text{ mit }a_i\in\Sigma
    \text{ und }u_i\in {L_a}_i
    \text{ für } i=1,\hdots,n\}\\
    &= \llang L \rrang \llang K \rrang
\end{align*}
\subsection*{b}
Sei $\hat{L} = \{w \in L^*| |w|=i\}$
\begin{align*}
    \llang L^* \rrang &= 
    \{w_1\hdots w_n |
    \exists w = a_1\hdots a_n \in L^*
    \text{ mit }a_i\in\Sigma
    \text{ und }w_i \in {L_a}_i
    \text{ für }i = 1,\hdots,n\}\\
    &=\{w_1\hdots w_n |
    \exists w = a_1\hdots a_n \in \underset{m\in\N}{\bigcup}\hat{L}_m
    \text{ mit }a_i\in\Sigma
    \text{ und }w_i \in {\underset{m\in\N}{\bigcup}{\hat{L}_m}_a}_i
    \text{ für }i = 1,\hdots,n\}\\
    &=
    \underset{m\in\N}{\bigcup} \{w_1\hdots w_n |
    \exists w = a_1\hdots a_n \in \hat{L}_m
    \text{ mit }a_i\in\Sigma
    \text{ und }w_i \in {{L_m}_a}_i
    \text{ für }i = 1,\hdots,n\}\\
    &=
    \llang L\rrang^*
\end{align*}
\subsection*{c}
Seien $L_1$ und $L_2$ zwei Reguläre Sprachen und $R_1$, $R_2$
reguläre Ausdrücke mit
$L(R_1)=L_1$ und $L(R_2)=L_2$
Dann ist $R_1 | R_2 = L(L_1 \cup L_2)$ auch Regulär.
Also ist
$\llang L \rrang = \underset{1 \leq i \leq n: a_i \in \Sigma, L}{\bigcup}{L_a}_i$
auch regulär
\subsection*{d}
Sei $\Sigma = \{a,b\}, L=\{a^nb^n\}, $L_a=\{a\}, L_b=\{a\}$
So sind $\llang L \rrang=\{a^{2n}\}, L_a, L_b$ regulär aber 
$L$ nicht.
\section*{2}
Seien $L_1$, $L_2$ reguläre Sprachen und $A_1$, $A_2$ DFAs mit
$L(A_1)=L_1, L(A_2)=L_2$.
Nun konstruieren wir den Produktautomaten $A_p$ von $A_1$ und $A_2$
Nun ändern wir die akzeptierten Zustandspaare $(q_1, q_2)$
zu denen wo $q_1 \in F_1 \land q_2 \notin F_2$ somit ist
$L(A_p) = L_1 \setminus L_2$ und $L_1 \setminus L_2$ eine
reguläre Sprache.
da $L_1 \cup L_2$ auch regulär (siehe 1.c) ist 
$L_1 \Delta L_2$ regulär.
\end{document}