\documentclass{article}
\usepackage{amsmath}
\usepackage{xypic}
\usepackage{listings}
\usepackage{amsmath}
\usepackage{amsfonts}
\title{Theo III - Blatt 4}
\author{Benjamin Möller \& Nick Daiber}

\begin{document}
\maketitle
\section*{3}
\subsection*{a}
Sei ein Graph $G(V, E)$, $c:E\rightarrow \mathbb R$, sowie $r:E\rightarrow \mathbb R$ und $s, t$ gegeben.
Um CSP auf das Rucksackproblem zu reduzieren geben wir
jedem Knoten $v\in V$ mit kanten $(u, v)$ einen Wert $w(v) = \frac{1}{\min\{u+c(u,v)\}}$.
Somit sind nahe knoten mehr wert als weit entfernte.
Danach hat jeder Knoten ein gewicht $w(v) = R(v)$.
Da das Rucksackproblem NP-Hart ist, ist auch CSP NP-Hart
\newpage
\subsection*{b}
\begin{lstlisting}[mathescape=true]
    func CSP(graph G(V, E), source, destination, R) {
        // this holds the shortest from the source to
        // node v with constraint $\leq R$ with all initially at $\infty$
        min_distance[v][R] = {{$\infty$, $\infty$, $\hdots$}, $\hdots$};

        min_distance[source][0] = 0;

        // iterating over all constraints
        for k = 0 to R {
            for u in V {
                // checking if u is reachable
                if (min_distance[u][k] < $\infty$) {
                    // checking all outgoing verteces
                    for each outgoing edge v {
                        weight = w(u, v)
                        // reassign if lower
                        if (k + weight $\leq$ R) {
                            min_distance[v][k + weight] = min(
                                min_distance[v][k + weight],
                                min_distance[u][k] + weight
                                )
                        }
                    }
                }
            }
        }

        // checking shortest path for all constraints
        for k = 0 to R {
            result = min(result, min_distance[target][k])
        }


        if (result == $\infty$)
            return NO_PATH_FOUND
        else
            return result
    }
\end{lstlisting}
\end{document}