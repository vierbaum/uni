\documentclass{article}
\usepackage{amsmath}
\usepackage{xypic}
\usepackage{listings}
\title{Theo III - Blatt 2}
\author{Benjamin Möller \& Nick Daiber}

\begin{document}
\maketitle
\section*{3}
\subsection*{1}
\paragraph{``$\Rightarrow$"\\}
wenn $D_i = d(i)$ gilt $d(1)=0=D_1$
\paragraph{``$\Leftarrow$"\\}
sei Sei $D_i = d(i)$ ein Distanzvektor mit $D_1\neq 0$,
dann ist $d(1) \neq 0$, aber der kürzeste weg $i\Rightarrow i$ ist immer 0.
Dies ist ein widerspruch zur Annahme
\subsection*{2}

Sei $D_i$ ein Distanzvektor, dann gilt nach Annahme
$d(i) = \inf\{c(\pi)\}$.
Da der Pfad $1\Rightarrow j \Rightarrow i \in\pi$
gitl $c(1\Rightarrow j)+\gamma(j,i)\in\{c(\pi)\}\Rightarrow \inf\{c(\pi)\}\geq c(1\Rightarrow j)+\gamma(j, i)$

\subsection*{3}
\begin{align*}
    d(j) = \inf\{c(1\Rightarrow j)\}\\
    \text{OBDA Wähle } i\in V\text{ so, dass}\\
    \inf\{c(1\Rightarrow j)\}=\inf\{c(1\Rightarrow i\Rightarrow j)\}\\
    \inf\{c(1\Rightarrow i)\} +  \gamma(i, j)\\
\end{align*}

\end{document}