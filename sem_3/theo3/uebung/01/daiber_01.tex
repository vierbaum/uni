\documentclass{article}
\usepackage{amsmath}
\usepackage{amssymb}
\usepackage{amsfonts}

\begin{document}
    \section{}
    \subsection{}
    Nach Annahme gilt
    \begin{align*}
        \limsup\limits_{n\rightarrow\infty}\frac{f(n)}{g(n)} =: A< \infty\\
        \Leftrightarrow\lim\limits_{n\rightarrow\infty}\sup(\frac{f(n)}{g(n)}) < \infty\\
        \limsup\limits_{n\rightarrow\infty}\frac{g(n)}{h(n)} =: B< \infty\\
        \Leftrightarrow\lim\limits_{n\rightarrow\infty}\sup(\frac{f(g)}{g(h)}) < \infty\\
    \end{align*}
    Es gilt also
    \begin{align*}
        \Leftrightarrow\lim\limits_{n\rightarrow\infty}\sup(\frac{f(n)}{g(n)}) \cdot
        \lim\limits_{n\rightarrow\infty}\sup(\frac{f(g)}{g(h)})\\
        = \lim\limits_{n\rightarrow\infty}\sup(\frac{f(n)}{g(n)})\cdot\sup(\frac{g(n)}{h(n)}) = A \cdot B\\
        \Leftrightarrow \lim\limits_{n\rightarrow\infty}\sup(\frac{f(n)}{g(n)}\cdot\frac{g(n)}{h(n)}) \leq A \cdot B\\
        \Leftrightarrow \lim\limits_{n\rightarrow\infty}\sup(\frac{f(n)}{h(n)}) \leq A \cdot B < \infty\\
        \Leftrightarrow \limsup\limits_{n\rightarrow\infty}\frac{f(n)}{h(n)} < \infty\\
        f\in O(h) \blacksquare
    \end{align*}
    \subsection{}
    Nach Annahme gilt
    \begin{align*}
        \limsup\limits_{n\rightarrow\infty}\frac{f_1(n)}{g_1(n)} < \infty\\
        \Leftrightarrow\lim\limits_{n\rightarrow\infty}\sup(\frac{f_1(n)}{g_1(n)}) < \infty\\
        \limsup\limits_{n\rightarrow\infty}\frac{f_2(n)}{g_2(n)} < \infty\\
        \Leftrightarrow\lim\limits_{n\rightarrow\infty}\sup(\frac{f_2(n)}{g_2(n)}) < \infty\\
    \end{align*}

    \begin{align*}
        \infty > f_1 \cdot f_2 &= \lim\limits_{n\rightarrow\infty}\sup(\frac{f_1}{g_1}) \cdot
        \lim\limits_{n\rightarrow\infty}\sup(\frac{f_2}{g_2})\\
        &\geq \lim\limits_{n\rightarrow\infty}\sup(\frac{f_1}{g_1}\cdot \frac{f_1}{g_2})\\
        &=\limsup\limits_{n\rightarrow\infty}\frac{f_1\cdot f_2}{g_1 \cdot g_2} < \infty\\
        &\Leftrightarrow f_1\cdot f_2\in O(g_1\cdot g_2) \blacksquare
    \end{align*}

    \newpage
    \section{}
    \subsection{}
    \begin{align*}
        &\limsup\limits_{n\rightarrow\infty}\frac{n^k}{a^n}\\
        &=\limsup\limits_{n\rightarrow\infty}\frac{k\cdot n^{(n-1)}}{\ln(a)a^n}\\
        \vdots\\
        &=\limsup\limits_{n\rightarrow\infty}\frac{\frac{2(k-1)}{2}n^0}{\ln^k(a)a^n}\\
        &=\limsup\limits_{n\rightarrow\infty}\frac{1}{a^n} =0\\
        &\Rightarrow n^k\in o(a^n)
    \end{align*}
    \subsection{}
    $\lim\limits_{n\rightarrow\infty}\frac{\log^rn}{n^l}=0$

    \section{}
    \subsection{}
    \paragraph{Induktionsvorraussetzung}
    \begin{equation*}
        \sum_{i=1}^1 i = 1 = \frac{1(1+1)}{2}
    \end{equation*}
    \paragraph{Induktionsannahme}
    \begin{equation*}
        \sum_{i=1}^n i = \frac{n(n+1)}{2}
    \end{equation*}
    \paragraph{Induktionsschritt}
    \begin{align*}
        &\sum_{i=1}^{n+1} i\\
        &=\sum_{i=1}^{n} i + n + 1\\
        &=\frac{n(n+1)}{2} + n + 1\\
        &=\frac{n(n+1)}{2} + \frac{2(n+1)}{2}\\
        &=\frac{(n+1)((n+1)+1)}{2}\\
    \end{align*}

    \subsection{}
    $n^{0.4}<\sqrt{n}<3n<n\log n<\sqrt{3}^n<2^n<n!$

    \section{}
    \subsection{f1}
    \begin{equation*}
        N=\sum_{i=1}^n\sum_{j=1}^n 1 = n^2
    \end{equation*}

    \subsection{f2}
    \begin{equation*}
        N=\sum_{i=1}^n\sum_{j=i}^n 1 = \frac{n(n+1)}{2} = \frac{n^2}{2}+\frac{n}{2}
    \end{equation*}

    \subsection{f3}
    \begin{equation*}
        N=\sum_{i=1}^n\sum_{j=1}^i 1 = \frac{n(n+1)}{2}
    \end{equation*}

    \subsection{f4}
    \begin{equation*}
        N=\sum_{i=1}^{1000n}\sum_{j=1}^{1000} 1 = 10^6n
    \end{equation*}

    \subsection{f5}
    \begin{equation*}
        N=\sum_{i=1}^{n-1}\sum_{i=1}^{n-2}\hdots=\prod_{i=1}^n i=(n-1)!
    \end{equation*}

    \subsection{f6}
    \begin{equation*}
        N=n(\log n+1)
    \end{equation*}

    \subsection{f7}
    \begin{equation*}
        N=2 \cdot n - 1
    \end{equation*}

    \subsection{f8}
    \begin{equation*}
        N=2 \cdot n - 1
    \end{equation*}

    \subsection{f9}
    \begin{equation*}
        N=\sum_1^{n}\sum_1^{n-1}\hdots = \prod_1^n i = n!
    \end{equation*}

    \subsection{f10}
    \begin{equation*}
        N=99999
    \end{equation*}

    \subsection{f11}
    \begin{equation*}
        N=\log n + 1
    \end{equation*}
    \subsection{f12}
    \begin{equation*}
        N = \frac{\log_4(n)}{2}
    \end{equation*}
\end{document}