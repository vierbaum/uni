\documentclass{article}
\usepackage{amsmath}
\title{Theo 3}
\author{Nick Daiber}

\begin{document}
\maketitle
\section{Sortieren}
Problem: Gegeben eine $n-$elementige Teilmenge $S = \{S_1, \hdots, S_n\}$ aus geordneten
Universum $U$.\\
Gesucht: Permutation $\pi: \{1, \hdots, n\} \rightarrow \{1, \hdots, n\}$, sodass gilt
$S_{\pi(1)}<S_{\pi(2)}<S_{\pi(3)}<$
\subsection{Ordnung}
\begin{itemize}
	\item{Buchstaben nach Position im Alphabet}
	\item{$w_1<w_2 \Leftrightarrow w_1 = s_1 x t_1, w_2=s_1 y t_2$\\
		mit $s_1 \in {a, \hdots, Z}^*$}
\end{itemize}

Sortierverfahren berechnet $\pi$
Aufwand:
\begin{itemize}
	\item{Zeit}
	\item{Platz}
	\item{Anzahl der Zufallsbits}
	\item{I/O operationen}
\end{itemize}

Analyse eines Algorithmus für beliebige Probleminstanzen oder mit speziellen Eigenschaften.\\
Wir betrachten nur beliebige Probleminstanzen.

\subsection{Bubble sort}
Eingabe: Array $A[1\hdots n]$(Zahlen)\\
Ausgabe: $A[]$ so sortiert, dass $\forall 1 \leq i \leq n-1 : A[i] \leq A[i+1]$
%TODO Programm
\subsubsection{ Sortiert Bubblesort A?}
\begin{itemize}
	\item{Alg verliert keine Zahl (weiß nicht, ob das wirklich bewiesen ist)}
\end{itemize}
Lemma: Sei i beliebig, betrachte $A[1, i]$ zu Zeitpunkt $1, 2$ 
zu 2 gilt $A[i] = max(g\in{1,\hdots,i}) A(g)$
$\Rightarrow$ das größte Element steht in $A[i]$ und kein Element geht verloren

Beweis: Idee - Sei $h$ der index in ${1,\hdots,i}$ mit $A[h] = max$
sobald $j=h$ wird $A[h]$ mit $A[h+1]$ getauscht, etc

\subsection{Laufzeit}
Uns interessiert die Laufzeit von BS
\begin{itemize}
	\item{Implementiere Alg und messen\\
		Problem: Andere Hardware, etc.}
	\item{Anzahl der Instruktionen zählen (im Code)}
	\item{Anzahl Vergleiche Zählen}
\end{itemize}
Behauptung: Abgesehen von Konstanten Faktor weicht das nicht ab vom Zählen aller 
ausgeführten Instruktionen

Annahme: Algorithmen haben konstante Größe

Lemma: Wenn eine Algorithmenbeschreibung c Zeilen hat, wird bei erfolgreichem Ablauf mindestens alle c 
Zeilen ein Vergleich ausgeführt

Vergleiche = n + $\sum_{l=2}^n l + \sum_{l=2}^{n-1} l =  n^2+n-2$

Bubblesort macht $n^2+n-1$ vgl.

Annahme: Rechner macht $10^9$ instruktionen pro sekunde
\end{document}
